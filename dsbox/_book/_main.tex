\documentclass[]{article}
\usepackage{lmodern}
\usepackage{amssymb,amsmath}
\usepackage{ifxetex,ifluatex}
\usepackage{fixltx2e} % provides \textsubscript
\ifnum 0\ifxetex 1\fi\ifluatex 1\fi=0 % if pdftex
  \usepackage[T1]{fontenc}
  \usepackage[utf8]{inputenc}
\else % if luatex or xelatex
  \ifxetex
    \usepackage{mathspec}
  \else
    \usepackage{fontspec}
  \fi
  \defaultfontfeatures{Ligatures=TeX,Scale=MatchLowercase}
\fi
% use upquote if available, for straight quotes in verbatim environments
\IfFileExists{upquote.sty}{\usepackage{upquote}}{}
% use microtype if available
\IfFileExists{microtype.sty}{%
\usepackage{microtype}
\UseMicrotypeSet[protrusion]{basicmath} % disable protrusion for tt fonts
}{}
\usepackage[margin=1in]{geometry}
\usepackage{hyperref}
\hypersetup{unicode=true,
            pdfborder={0 0 0},
            breaklinks=true}
\urlstyle{same}  % don't use monospace font for urls
\usepackage{natbib}
\bibliographystyle{plainnat}
\usepackage{color}
\usepackage{fancyvrb}
\newcommand{\VerbBar}{|}
\newcommand{\VERB}{\Verb[commandchars=\\\{\}]}
\DefineVerbatimEnvironment{Highlighting}{Verbatim}{commandchars=\\\{\}}
% Add ',fontsize=\small' for more characters per line
\usepackage{framed}
\definecolor{shadecolor}{RGB}{248,248,248}
\newenvironment{Shaded}{\begin{snugshade}}{\end{snugshade}}
\newcommand{\KeywordTok}[1]{\textcolor[rgb]{0.13,0.29,0.53}{\textbf{#1}}}
\newcommand{\DataTypeTok}[1]{\textcolor[rgb]{0.13,0.29,0.53}{#1}}
\newcommand{\DecValTok}[1]{\textcolor[rgb]{0.00,0.00,0.81}{#1}}
\newcommand{\BaseNTok}[1]{\textcolor[rgb]{0.00,0.00,0.81}{#1}}
\newcommand{\FloatTok}[1]{\textcolor[rgb]{0.00,0.00,0.81}{#1}}
\newcommand{\ConstantTok}[1]{\textcolor[rgb]{0.00,0.00,0.00}{#1}}
\newcommand{\CharTok}[1]{\textcolor[rgb]{0.31,0.60,0.02}{#1}}
\newcommand{\SpecialCharTok}[1]{\textcolor[rgb]{0.00,0.00,0.00}{#1}}
\newcommand{\StringTok}[1]{\textcolor[rgb]{0.31,0.60,0.02}{#1}}
\newcommand{\VerbatimStringTok}[1]{\textcolor[rgb]{0.31,0.60,0.02}{#1}}
\newcommand{\SpecialStringTok}[1]{\textcolor[rgb]{0.31,0.60,0.02}{#1}}
\newcommand{\ImportTok}[1]{#1}
\newcommand{\CommentTok}[1]{\textcolor[rgb]{0.56,0.35,0.01}{\textit{#1}}}
\newcommand{\DocumentationTok}[1]{\textcolor[rgb]{0.56,0.35,0.01}{\textbf{\textit{#1}}}}
\newcommand{\AnnotationTok}[1]{\textcolor[rgb]{0.56,0.35,0.01}{\textbf{\textit{#1}}}}
\newcommand{\CommentVarTok}[1]{\textcolor[rgb]{0.56,0.35,0.01}{\textbf{\textit{#1}}}}
\newcommand{\OtherTok}[1]{\textcolor[rgb]{0.56,0.35,0.01}{#1}}
\newcommand{\FunctionTok}[1]{\textcolor[rgb]{0.00,0.00,0.00}{#1}}
\newcommand{\VariableTok}[1]{\textcolor[rgb]{0.00,0.00,0.00}{#1}}
\newcommand{\ControlFlowTok}[1]{\textcolor[rgb]{0.13,0.29,0.53}{\textbf{#1}}}
\newcommand{\OperatorTok}[1]{\textcolor[rgb]{0.81,0.36,0.00}{\textbf{#1}}}
\newcommand{\BuiltInTok}[1]{#1}
\newcommand{\ExtensionTok}[1]{#1}
\newcommand{\PreprocessorTok}[1]{\textcolor[rgb]{0.56,0.35,0.01}{\textit{#1}}}
\newcommand{\AttributeTok}[1]{\textcolor[rgb]{0.77,0.63,0.00}{#1}}
\newcommand{\RegionMarkerTok}[1]{#1}
\newcommand{\InformationTok}[1]{\textcolor[rgb]{0.56,0.35,0.01}{\textbf{\textit{#1}}}}
\newcommand{\WarningTok}[1]{\textcolor[rgb]{0.56,0.35,0.01}{\textbf{\textit{#1}}}}
\newcommand{\AlertTok}[1]{\textcolor[rgb]{0.94,0.16,0.16}{#1}}
\newcommand{\ErrorTok}[1]{\textcolor[rgb]{0.64,0.00,0.00}{\textbf{#1}}}
\newcommand{\NormalTok}[1]{#1}
\usepackage{longtable,booktabs}
\usepackage{graphicx,grffile}
\makeatletter
\def\maxwidth{\ifdim\Gin@nat@width>\linewidth\linewidth\else\Gin@nat@width\fi}
\def\maxheight{\ifdim\Gin@nat@height>\textheight\textheight\else\Gin@nat@height\fi}
\makeatother
% Scale images if necessary, so that they will not overflow the page
% margins by default, and it is still possible to overwrite the defaults
% using explicit options in \includegraphics[width, height, ...]{}
\setkeys{Gin}{width=\maxwidth,height=\maxheight,keepaspectratio}
\IfFileExists{parskip.sty}{%
\usepackage{parskip}
}{% else
\setlength{\parindent}{0pt}
\setlength{\parskip}{6pt plus 2pt minus 1pt}
}
\setlength{\emergencystretch}{3em}  % prevent overfull lines
\providecommand{\tightlist}{%
  \setlength{\itemsep}{0pt}\setlength{\parskip}{0pt}}
\setcounter{secnumdepth}{5}
% Redefines (sub)paragraphs to behave more like sections
\ifx\paragraph\undefined\else
\let\oldparagraph\paragraph
\renewcommand{\paragraph}[1]{\oldparagraph{#1}\mbox{}}
\fi
\ifx\subparagraph\undefined\else
\let\oldsubparagraph\subparagraph
\renewcommand{\subparagraph}[1]{\oldsubparagraph{#1}\mbox{}}
\fi

%%% Use protect on footnotes to avoid problems with footnotes in titles
\let\rmarkdownfootnote\footnote%
\def\footnote{\protect\rmarkdownfootnote}

%%% Change title format to be more compact
\usepackage{titling}

% Create subtitle command for use in maketitle
\newcommand{\subtitle}[1]{
  \posttitle{
    \begin{center}\large#1\end{center}
    }
}

\setlength{\droptitle}{-2em}
  \title{}
  \pretitle{\vspace{\droptitle}}
  \posttitle{}
  \author{}
  \preauthor{}\postauthor{}
  \date{}
  \predate{}\postdate{}

\usepackage{booktabs}
\usepackage{amsthm}
\makeatletter
\def\thm@space@setup{%
  \thm@preskip=8pt plus 2pt minus 4pt
  \thm@postskip=\thm@preskip
}
\makeatother

\begin{document}

{
\setcounter{tocdepth}{2}
\tableofcontents
}
\section{Introduction}\label{intro}

This is not a book per se, at least not yet. It's a place for organizing
materials for teaching (and learning) data science with R, RStudio, the
tidyverse and tidyverse friendly packages. It's called \textbf{Data
Science Course in a Box}, as it contains all the materials you (an
educator) might need to teach data science or you (a learner) might find
useful to learn about them.

\subsection{Who is this course for?}\label{who-is-this-course-for}

The materials in this box are designed for learners who have no
background in data science, statistics, or programming. However, they
also assume that the learners are interested in making sense of
(sometimes messy) data and willing to dive into the documentation of the
packages we introduce.

\subsection{What is in the box?}\label{what-is-in-the-box}

\begin{itemize}
\tightlist
\item
  Slides
\item
  Labs
\item
  Assignments
\item
  Exams
\item
  Infrastructure guide
\end{itemize}

\subsection{How is the course content
organized?}\label{how-is-the-course-content-organized}

\begin{itemize}
\tightlist
\item
  Part 1: Exploring data - wrangle + visualize + collect
\item
  Part 2: Making rigorous conclusions - modeling + inference
\item
  Part 3: Looking forward - \ldots{}
\end{itemize}

\subsection{Prerequisites}\label{prerequisites}

There are four things you need to run the code in this book: R, RStudio,
a collection of R packages called the \textbf{tidyverse}, and a handful
of other packages. Packages are the fundamental units of reproducible R
code. They include reusable functions, the documentation that describes
how to use them, and sample data.

\subsubsection{On the Cloud}\label{on-the-cloud}

You can access all on the cloud, via
\href{http://rstudio.cloud/}{RStudio Cloud}, and avoid local
installation. {[}TO DO: ADD LINK TO CLOUD DSBOX WORKSPACE{]}

\subsubsection{Locally}\label{locally}

\paragraph{R}\label{r}

To download R, go to CRAN, the \textbf{c}omprehensive \textbf{R}
\textbf{a}rchive \textbf{n}etwork. CRAN is composed of a set of mirror
servers distributed around the world and is used to distribute R and R
packages. Don't try and pick a mirror that's close to you: instead use
the cloud mirror, \url{https://cloud.r-project.org}, which automatically
figures it out for you.

A new major version of R comes out once a year, and there are 2-3 minor
releases each year. It's a good idea to update regularly. Upgrading can
be a bit of a hassle, especially for major versions, which require you
to reinstall all your packages, but putting it off only makes it worse.

\subsubsection{RStudio}\label{rstudio}

{[}TO DO: THERE ARE SOME WORDS BORROWED FROM R4DS BELOW, CLEAN THEM
UP.{]}

RStudio is an integrated development environment, or IDE, for R
programming. Download and install it from
\url{http://www.rstudio.com/download}. RStudio is updated a couple of
times a year. When a new version is available, RStudio will let you
know. It's a good idea to upgrade regularly so you can take advantage of
the latest and greatest features.

\subsubsection{The tidyverse}\label{the-tidyverse}

You'll also need to install some R packages. An R \textbf{package} is a
collection of functions, data, and documentation that extends the
capabilities of base R. Using packages is key to the successful use of
R. The majority of the packages that you will learn in this book are
part of the so-called tidyverse. The packages in the tidyverse share a
common philosophy of data and R programming, and are designed to work
together naturally.

You can install the complete tidyverse with a single line of code:

\begin{Shaded}
\begin{Highlighting}[]
\KeywordTok{install.packages}\NormalTok{(}\StringTok{"tidyverse"}\NormalTok{)}
\end{Highlighting}
\end{Shaded}

On your own computer, type that line of code in the console, and then
press enter to run it. R will download the packages from CRAN and
install them on to your computer. If you have problems installing, make
sure that you are connected to the internet, and that
\url{https://cloud.r-project.org/} isn't blocked by your firewall or
proxy.

You will not be able to use the functions, objects, and help files in a
package until you load it with \texttt{library()}. Once you have
installed a package, you can load it with the \texttt{library()}
function:

\begin{Shaded}
\begin{Highlighting}[]
\KeywordTok{library}\NormalTok{(tidyverse)}
\CommentTok{#> -- Attaching packages ----------------------------------------------------------- tidyverse 1.2.1 --}
\CommentTok{#> √ ggplot2 2.2.1          √ purrr   0.2.4     }
\CommentTok{#> √ tibble  1.4.2          √ dplyr   0.7.4.9002}
\CommentTok{#> √ tidyr   0.8.0          √ stringr 1.3.0     }
\CommentTok{#> √ readr   1.1.1          √ forcats 0.3.0}
\CommentTok{#> -- Conflicts -------------------------------------------------------------- tidyverse_conflicts() --}
\CommentTok{#> x dplyr::filter() masks stats::filter()}
\CommentTok{#> x dplyr::lag()    masks stats::lag()}
\end{Highlighting}
\end{Shaded}

This tells you that tidyverse is loading the ggplot2, tibble, tidyr,
readr, purrr, and dplyr packages. These are considered to be the
\textbf{core} of the tidyverse because you'll use them in almost every
analysis.

Packages in the tidyverse change fairly frequently. You can see if
updates are available, and optionally install them, by running
\texttt{tidyverse\_update()}.

\paragraph{Other packages}\label{other-packages}

There are many other excellent packages that are not part of the
tidyverse, because they solve problems in a different domain, or are
designed with a different set of underlying principles. This doesn't
make them better or worse, just different. In other words, the
complement to the tidyverse is not the messyverse, but many other
universes of interrelated packages. As you tackle more data science
projects with R, you'll learn new packages and new ways of thinking
about data.

In this book we'll use three data packages from outside the tidyverse:

\begin{Shaded}
\begin{Highlighting}[]
\KeywordTok{install.packages}\NormalTok{(}\KeywordTok{c}\NormalTok{(}\StringTok{"nycflights13"}\NormalTok{, }\StringTok{"fivethirtyeight"}\NormalTok{))}
\end{Highlighting}
\end{Shaded}

{[}TO DO: ADD OTHERS{]}

These packages provide data on airline flights, world development, and
baseball that we'll use to illustrate key data science ideas.

\part{Exploring data}\label{part-exploring-data}

\section{Introduction}\label{explore-intro}

This is where into to part 1 goes.

\part{Making rigorous
conclusions}\label{part-making-rigorous-conclusions}

\section{Introduction}\label{rigor-intro}

This is where into to part 2 goes.

\part{Looking forward}\label{part-looking-forward}

\section{Introduction}\label{forward-intro}

This part has a bunch of independent modules, each on a different topic.
Pick and choose to your liking, depending on how much time you have to
cover them.

\part{Infrastructure}\label{part-infrastructure}

\section{Introduction}\label{infra-intro}

Intro to part 4 goes here.


\end{document}
